\documentclass[pdftex,12pt,letterpaper]{article}

\usepackage[letterpaper,margin=1.0in]{geometry}
\usepackage[pdftex]{graphicx}
\usepackage{amsmath}
\usepackage{amsfonts}
\usepackage{amsthm}
\usepackage{hyperref}
\usepackage{longtable}
\usepackage{algorithm}
\usepackage{algpseudocode}

\newcommand{\HRule}{\rule{\linewidth}{0.5mm}}
\newcommand{\eqdef}{\overset{\text{def}}{=}}
\newcommand{\sign}{\text{\upshape{sign}}}
\newcommand{\suchthat}{\,|\,}

\newtheorem{proposition}{Proposition}
\newtheorem{lemma}{Lemma}
\newtheorem{conjecture}{Conjecture}
\newtheorem{definition}{Definition}

\begin{document}
\title{Humps notes}
\date{}
\maketitle

Nullclines of $\sigma(Wv)-v$:
\begin{align}
\sigma(W_{i,:}v) - v_i &= 0 \\
W_{i,j\ne i}v_{j\ne i} &= \sigma^{-1}(v_i) - W_{i,i}v_i
\end{align}
Suggests a function $\zeta_i(x) = \sigma^{-1}(x) - W_{i,i}x$.  Implies that nullclines are ``ruled hypersurfaces'': they are a disjoint union of linear spaces, one linear space for each $v_i$.  The $(N-2)$-dimensional linear space for a fixed $v_i = a$ is defined by
\begin{align}
\left[\begin{array}{c} \hat{w}^T \\ I_{i,:}\end{array}\right]v = \left[\begin{array}{c} \zeta_i(a) \\ a\end{array}\right]
\end{align}
where $\hat{w}^T = W_{i,:} - W_{i,i}I_{i,:}$. (proof of disjoint union TODO)

Degenerate case when $\hat{w} = \mathbf{0}^T$:  then only solutions when $\zeta_i(a) = 0$.  Denote the solutions by $\{-a^*, 0, a^*\}$.  Then the nullcline is three disconnected, axis orthogonal hyperplanes satisfying $v_i \in \{-a^*, 0, +a^*\}$. (Humps go to infinity).

``Central points'' on the $i^{th}$ nullcline, for a given $a$: the closest point to $aI_{i,:}^T$.  Of the form $\hat{v} = aI_{i,:}^T + s\hat{w}$ for some $s$.
\begin{align}
\hat{w}^T\hat{v} = \zeta_i(a) \\
\hat{w}^T(aI_{i,:}^T + s\hat{w}) = \zeta_i(a) \\
s\hat{w}^T\hat{w} = \zeta_i(a) \\
s = \zeta_i(a) / (\hat{w}^T\hat{w})
\end{align}

Distance to central points from axis is then calculated to be $|| s\hat{w} || = |\zeta_i(a)| / ||\hat{w}||$, with extremes at
\begin{align}
\zeta_i'(\tilde{a}) = 1 / (1 - \tilde{a}^2) - W_{i,i} = 0 \\
\tilde{a} = \pm \sqrt{1 - 1/W_{i,i}} \\
\end{align}
whenever $W_{i,i} > 1$.  These extremities are refered to as ``humps''.  Here, the distance $|\zeta_i(\tilde{a})| / ||\hat{w}||$ is large when
\begin{align}
|\zeta_i(\tilde{a})|/||\hat{w}||
&= |\sigma^{-1}(\sqrt{1 - 1/W_{i,i}}) - W_{i,i}\sqrt{1 - 1/W_{i,i}}|/||\hat{w}||
\end{align}
is large.  In particular, for fixed diagonal, rescaling the off-diagonals can make the humps arbitrarily large or small.

See figure of \texttt{humps.m}.

\end{document}

